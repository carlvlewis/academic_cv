%%%%%%%%%%%%%%%%%%%%%%%%%%%%%%%%%%%%%%%%%
% Compact Academic CV
% LaTeX Template
% Version 1.0 (10/6/2012)
%
% This template has been downloaded from:
% http://www.LaTeXTemplates.com
%
% Original author:
% Dario Taraborelli (http://nitens.org/taraborelli/home)
%
% License:
% CC BY-NC-SA 3.0 (http://creativecommons.org/licenses/by-nc-sa/3.0/)
%
% Important:
% This template needs to be compiled using XeLaTeX
%
% Note: this template has the option to use the Hoefler Text font, see the
% font configurations section below for instructions on using this font
%
%%%%%%%%%%%%%%%%%%%%%%%%%%%%%%%%%%%%%%%%%

%----------------------------------------------------------------------------------------
%	PACKAGES AND OTHER DOCUMENT CONFIGURATIONS
%----------------------------------------------------------------------------------------

\documentclass[11pt, letterpaper]{article} % Document font size and paper size

\usepackage{fontspec} % Allows the use of OpenType fonts

\usepackage{geometry} % Allows the configuration of document margins
\geometry{a4paper, textwidth=5.5in, textheight=8.5in, marginparsep=7pt, marginparwidth=.6in} % Document margin settings
\setlength\parindent{0in} % Remove paragraph indentation

\usepackage[usenames,dvipsnames]{color} % Custom colors

\usepackage{sectsty} % Allows changing the font options for sections in a document
\usepackage[normalem]{ulem} % Custom underlining
\usepackage{xunicode} % Allows generation of unicode characters from accented glyphs
\defaultfontfeatures{Mapping=tex-text} % Converts LaTeX specials (``quotes'' --- dashes etc.) to unicode

\usepackage{marginnote} % For margin years
\newcommand{\years}[1]{\marginnote{\scriptsize #1}} % New command for including margin years
\renewcommand*{\raggedleftmarginnote}{}
\setlength{\marginparsep}{7pt} % Slightly increase the distance of the margin years from the contant
\reversemarginpar

\usepackage[xetex, bookmarks, colorlinks, breaklinks, pdftitle={Albert Einstein - vita},pdfauthor={Albert Einstein}]{hyperref} % PDF setup - set your name and the title of the document to be incorporated into the final PDF file meta-information
\hypersetup{linkcolor=blue,citecolor=blue,filecolor=black,urlcolor=MidnightBlue} % Link colors

%----------------------------------------------------------------------------------------
%	FONT CONFIGURATIONS
%----------------------------------------------------------------------------------------

% Two font choices are available in this template, the default is Linux Libertine, available for free at: http://www.linuxlibertine.org while the secondary choice is Hoefler Text which comes bundled with Mac OSX.
% To use Hoefler Text, comment out the Linux Libertine block below and uncomment the Hoefler Text block. You will also need to replace the "\&" characters with "\amper{}" in section titles.

% Linux Libertine Font (default)
%\setromanfont [Ligatures={Common}, Numbers={OldStyle}, Variant=01]{Linux Libertine O} % Main text font
%%\setmonofont[Scale=0.8]{Monaco} % Set mono font (e.g. phone numbers)
%\sectionfont{\mdseries\upshape\Large} % Set font options for sections
%\subsectionfont{\mdseries\scshape\normalsize} % Set font options for subsections
%\subsubsectionfont{\mdseries\upshape\large} % Set font options for subsubsections
%\chardef\&="E050 % Custom ampersand character

% Hoefler Text Font (bundled with Mac OSX)
\setromanfont [Ligatures={Common}, Numbers={OldStyle}]{Hoefler Text} % Main text font
\setmonofont[Scale=0.8]{Monaco} % Set mono font (e.g. phone numbers)
\setsansfont[Scale=0.9]{Optima Regular} % Set sans font, used in the main name and titles in the document
\newcommand{\amper}{{\fontspec[Scale=.95]{Hoefler Text}\selectfont\itshape\&}} % Custom ampersand character
\sectionfont{\sffamily\mdseries\large\underline} % Set font options for sections
\subsectionfont{\rmfamily\mdseries\scshape\normalsize} % Set font options for subsections
\subsubsectionfont{\rmfamily\bfseries\upshape\normalsize} % Set font options for subsubsections

%----------------------------------------------------------------------------------------

\begin{document}



%----------------------------------------------------------------------------------------
%	CONTACT AND GENERAL INFORMATION SECTION
%----------------------------------------------------------------------------------------

{\LARGE Karl Pichotta}\\[1cm] % Your name
Department of Computer Science \\
  University of Texas at Austin \\
  2317 Speedway, 2.302 \\
	Austin, Texas 78712 \\
U.S.A.\\[.2cm]
%Phone: \texttt{609-734-8000}\\ % Your phone number
%Fax: \texttt{609-924-8399}\\[.2cm] % Your fax number
Email: \href{mailto:pichotta@cs.utexas.edu}{pichotta@cs.utexas.edu}\\ % Your email address
\textsc{url}: \url{http://cs.utexas.edu/~pichotta}\\

%\vfill % Whitespace between contact information and specific CV information


%------------------------------------------------

\section*{Education}

  \years{}Ph.D (in progress), Computer Science, University of
  Texas at Austin (2016, expected). \\
  \years{2013}MS, Computer Science, University of Texas at Austin. \\
  \years{2008}BS, Symbolic Systems (Honors), Minor in Mathematics, Stanford
  University..



\section*{Research Interests}

Natural Language Processing,
Document and Discourse-level Computational Semantics,
Machine Learning.


\section*{Publications}

\subsection*{Journal Articles}

\years{2012}Vladimir Lifschitz, Karl Pichotta and Fangkai Yang. Relational
Theories with Null Values and Non-Herbrand Stable Models. \textit{Theory and
Practice of Logic Programming},  12(4-5):565-582. 2012.

\subsection*{Conference Proceedings}

\years{2014}Karl Pichotta and Raymond J.\ Mooney. Statistical Script Learning with Multi-Argument Events.
\textit{To Appear in the Proceedings of the 14th Conference of the European Chapter of the Association for Computational Linguistics (EACL 2014)}.

\years{2013}Karl Pichotta and John DeNero. Identifying Phrasal Verbs Using Many
Bilingual Corpora. \textit{Proceedings of the 2013 Conference on Empirical
Methods in Natural Language Processing (EMNLP 2013)}.




\subsection*{Other Publications}
\years{2008}Karl Pichotta. Processing Paraphrases and Phrasal Implicatives in
the Bridge Question-Answering System.
Undergraduate Honors Thesis, Symbolic Systems Program, Stanford University.
2008.





\section*{Honors, Awards, \& Fellowships}

\years{2010}Microelectronics and Computer Development (MCD) Fellowship, University of Texas at Austin. \\
\years{2006}Summer Research Fellowship, Stanford University. \\
\years{2004} Robert C. Byrd Honors Scholarship. \\
\years{2004}National Merit Scholarship.





\section*{Talks}

%\years{2011} ``Modern Applications of Artificial Intelligence.'' Guest Lecture,
%CS303E: Elements of Computers and Programming, University of Texas at Austin. \\
\years{2010}``Advanced Speech Recognition Techniques and Experiences.'' Panel
		Discussion, SpeechTEK Europe Conference, London.


\section*{Teaching}

\subsection*{Stanford University}

\years{2006-2008}Section Leader, Programming Methodology \& Programming Abstractions: Fall 2006--Spring 2008.




\section*{Research Positions}

\years{2012--2013}Defense Advanced Research Projects Agency, Deep Exploration and Filtering of Text (DEFT) Program, Graduate Research Assistant.\\
\years{2012}Google, Research Intern (Google Translate).\\
\years{2011--2012}Defense Advanced Research Projects Agency, Machine Reading Project (MRP) Program, Graduate Research Assistant.\\
\years{2008}SRI Artificial Intelligence Center, Student Associate.\\
\years{2007}PARC (Palo Alto Research Center) Natural Language Theory and Technology Group, Research Intern. \\
\years{2006}Stanford University Electrical Engineering Department, Research Assistant.



\section*{Other Experience}

\years{2008--2010}Versay Solutions, Software Engineer.\\
\years{2004,2005} Motorola, Engineering Intern.




\section*{Professional Activities}
\years{2013}Member, Association for Computational Linguistics.\\
\years{2009}Member, Association for Symbolic Logic.\\
\years{2008}Secondary Reviewer,
ICAPS'08 Scheduling and Planning Applications Workshop.

%\section*{Departmental Service}
%\years{2012} Forum for Artificial Intelligence coordinator.\\
%\years{2012} Deep Exploration and Filtering of Text (DEFT) reading group coordinator.\\
%\years{2011}Machine Reading Project (MRP) reading group coordinator.


\section*{Languages}
\years{}English (native).\\
\years{}Spanish (conversational).\\
\years{}French (good). \\
\years{}Sanskrit (can read with dictionary).\\
\years{}Ancient Greek (can read with dictionary).






\vfill{} % Whitespace before final footer

%----------------------------------------------------------------------------------------
%	FINAL FOOTER
%----------------------------------------------------------------------------------------

\begin{center}
{\scriptsize Last updated: \today }
\end{center}

%----------------------------------------------------------------------------------------

\end{document}
